	% LaTeX file for resume 
% This file uses the resume document class (res.cls)

\documentclass[10pt,stdletter,dateno]{newlfm}
%\usepackage{kpfonts}
\usepackage{url}
\usepackage{hyperref}
\hypersetup{
colorlinks=true,
linkcolor=blue,
filecolor=magenta,      
urlcolor=blue,
}
\usepackage{csquotes}

\widowpenalty=1000
\clubpenalty=1000

\newlfmP{headermarginskip=20pt}
\newlfmP{sigsize=50pt}
\newlfmP{dateskipafter=20pt}
\newlfmP{addrfromphone}
\newlfmP{addrfromemail}
\PhrPhone{Phone}
\PhrEmail{Email}

\namefrom{Rahulkumar\ Gayatri}
\addrfrom{%
    \today\\[10pt]
601, Dream Home Palace\\
Hyderabad, Telangana\\
India, 500035
}
\phonefrom{+91-7899423040}
\emailfrom{rahulgayatri84@gmail.com}

\greetto{To Whom It May Concern,}
\closeline{Sincerely,}
\begin{document}
\begin{newlfm}
	   This is a cover letter for a PostDoc position in compilation for Heterogeneous Systems at the Uppsala Universitet.
	   My qualifications include a Ph.D. Degree from Universitat Politecnica de Catalunya, Barcelona, Spain. 
	   During this period I was a part of the Programming Models group at the Barcelona Supercomputing Center (BSC). 
	   My advisors were Rosa M. Badia and Eduard Ayguade. 
	   My Doctoral thesis was focused on speculative synchronization techniques for the StarSs framework, a task-based programming model.
	   My work included the development of StarSs compiler and runtime features that handle synchronization of multiple threads.
	   I used the Transactional Memory framework to achieve the thread synchronization in StarSs.
	   As a part of the programming models group, I was also involved in the design and implementation of parallel algorithms that use the StarSs framework.
	   Currently I am working as a Technical Specialist in High Performance Computing Group of Wipro Infotech. 
%
	   \par
	   My work has been published in \textit{EuroPar 2012} and \textit{HiPC 2013} conferences as well as in \textit{MULTIPROG, 2015} workshop.  
	   I am also the co-author of a journal paper published in \textit{Microprocessors and Microsystems, 2014} which explains the work accomplished as a part of the Teraflux project which was sponsored and supported by the European Union. 
	   My resume contains the information of all my publications and their references. 
%
	   \par
	   In my current job, I provide technical assistance to clients who wish to parallelize their code and take advantage of their multicore infrastructure. 
	   Depending on their requirement, I collaborate with them on the design, implementation or on both to help them improve the performance of their application/algorithm.
	   This implies that I constantly work in collaboration with the clients and improve upon the existing solution.
	   It requires understanding the critical points made by the clients and taking positives to improve the work being done.


	   
%	   
	   \par
	   


%
\par
	   My background in mathematics helps me develop novel ideas for parallelization of algorithms and efficient ways to implement them. 
	   That combined with my knowledge of parallel programming models and my current project makes my skill set match the requirements of the job posting.
	   As a part of the programming models group at Barcelona Supercomputing Center (\url{http://www.bsc.es/computer-sciences/programming-models}), I have also learned to effectively collaborate with other engineers to achieve the set goals. 
	   Given my experience and background, I feel my skill set matches the requirements of the job. 
	   Additionally, I feel the position gives me a platform to solve interesting and challenging problems and hence I request you to consider my application. 
%
\par
Please let me know if there are any other materials or information that will assist you in processing my application. 
The earliest I can start working at the JSC will be in June, 2016. 
Thank you for your consideration. I look forward to hearing from you.
%
%	   The multidisciplinary and challenging area of parallel computing has taught me to find innovative and efficient solutions to a given problem. 
%	   Work in R\&D did not only provide me with an in depth-knowledge in the areas of mathematics, engineering and CS, but has also given me a valuable insight in the areas of problem analysis, solution design and rapid prototyping.
%	   I am also a recipient of the FI-AGAUR scholarship from the Catalan government which is a grant given to encourage research.  
%
\end{newlfm}
\end{document}
