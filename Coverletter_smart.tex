	% LaTeX file for resume
% This file uses the resume document class (res.cls)

\documentclass[10pt,stdletter,dateno]{newlfm}
%\usepackage{kpfonts}
\usepackage{url}
\usepackage{hyperref}
\hypersetup{
colorlinks=true,
linkcolor=blue,
filecolor=magenta,
urlcolor=blue,
}
\usepackage{csquotes}

\widowpenalty=1000
\clubpenalty=1000

\newlfmP{headermarginskip=20pt}
\newlfmP{sigsize=50pt}
\newlfmP{dateskipafter=20pt}
\newlfmP{addrfromphone}
\newlfmP{addrfromemail}
\PhrPhone{Phone}
\PhrEmail{Email}

\namefrom{Rahulkumar\ Gayatri}
\addrfrom{%
    \today\\[10pt]
601, Dream Home Palace\\
Hyderabad, Telangana\\
India, 500035
}
\phonefrom{+91-7899423040}
\emailfrom{rahulgayatri84@gmail.com}

\greetto{To Whom It May Concern,}
\closeline{Sincerely,}
\begin{document}
\begin{newlfm}
	   This is a cover letter for the position of {\it Senior Software Engineer} for the development of the SimMobility Short and Mid Term activity based simulation software.
	   My name is Dr. Rahulkumar Gayatri and I am currently based in Bangalore, India.
	   I work as a technical specialist in the High Performance Computing group (HPC) of Wipro Infotech.
	   I have been in this position for the last six months.
	   My major responsibilities include providing assistance and the necessary technical help to clients who wish to parallelize their application/algorithm.
	   The level of my involvement depends on the client's need. I collaborate at design, implementation and analysis or all the stages.
	   The technologies I use depend on the infrastructure of the client. I use the Pthreads and the OpenMP programming models more frequently to parallelize the code.
	   I am also experienced with the MPI, OMPSs and STM-based programming models.
%
	   \par
	   Currently I am working on a project called Moose, which involves simulation of neurons in human brain.
	   The project was designed and implemented at the National Center for Biological Sciences, India (NCBS).
	   The multiscale characteristic of this project allows both distributed and shared memory parallelization to be introduced to speedup the simulation.
	   At present, my focus is on the thread-based implementation using pthreads, where each thread updates a part of the cell structure at every time-step.
	   Later on, the project will include simultaneous updates of different cells on different nodes, which in turn will use threaded parallelism within the cell.
%
	   \par
	   Previous to this, I was a Doctoral student at Barcelona Supercomputing Center (BSC), Barcelona, Spain. I graduated in March, 2015.
	   My advisors were Rosa M.Badia and Eduard Ayguade. Their contact information is included in my resume.
	   During this period, I was a part of the Programming Models group at the Barcelona Supercomputing Center (BSC).
	   My Doctoral thesis was focused on speculative synchronization techniques for the StarSs framework, a task-based programming model.
	   Synchronization of threads in a shared memory muti-core processor is necessary during critical memory update and to maintain sequential consistency of the parallel code.
%
	   \par
	   Speculative synchronization using Software Transactional Memory (STM) based approach was adopted instead of the traditional locks and mutexes to achieve concurrency.
	   I extended the StarSs framework with compiler directives and the associated runtime support for concurrency control among multiple threads.
	   The results achieved with this implementation showed an increase in performance of applications which have high contention for locks.
	   Also the speculative nature of STM increases the available parallelism in an application.
%
	   \par
	   An extension to speculative memory updates is speculative execution of tasks.
	   Compiler and runtime assistance was added to the StarSs framework where tasks can be scheduled before their presence in the execution flow can be confirmed.
	   Such speculatively generated tasks are executed as transactions so that their updates can be rolled back in case the speculation fails.
	   To avoid the overhead associated with an STM-framework, such as conflict detection and concurrency control, I implemented a lighweight rollback mechanism which can be used to undo the updates of tasks.
	   The idea of greedy task execution, wherein the tasks are scheduled even before their validity can be ascertained has given an average of 20\% performance benefits.
	   This is a highly positive result, since the performance improvement was achieved on an already parallelized code.
%
	   \par
	   Also at BSC, I worked on porting applications using SMPSs, StarSs implementation for SMPs onto Symmetric Multiprocessors (SMPs).
	   I built applications that show the performance benefits, ease of use and portability of using the StarSs framework.
	   The applications are a part of th StarSs application repository.
	   My background in mathematics helps me develop novel ideas for parallelization of algorithms and efficient ways to implement them.
	   In my Mtech thesis, I designed and implemented a Breadth First Search algorithm, that optimises the use of low memory available in the Synergistic Processing Ellement(SPE) of IBM's Cell.B.E processor.
	   The strong point of the implementation was the design of a data structure to represent node. This data structure helped in increasing memory locality.
%
	   \par
	   My doctoral work has been published in \textit{EuroPar 2012} and \textit{HiPC 2013} conferences as well as in \textit{MULTIPROG, 2015} workshop.
	   I am also the co-author of a journal paper published in \textit{Microprocessors and Microsystems, 2014} which explains the work accomplished as a part of the Teraflux project which was sponsored and supported by the European Union.
	   I have 2 posters which highlight my work on parallelization of applications.
	   My resume contains the information of all my publications.
%
	   \par
	   The implementations of the above mentioned ideas and parallel applications were done using the C and C++ programming language.
	   My current job as the lead technical consultant for the HPC group has given me exposure to the software project cycle.
	   As a part of the programming models group at Barcelona Supercomputing Center (\url{http://www.bsc.es/computer-sciences/programming-models}), I have also learned to effectively collaborate with other engineers to achieve the set goals.
	   Working with the TERAFLUX group gave me the experience of collaborating with researchers and engineers who are at different physical locations.
	   %
	   \par
	   The above mentioned details i.e., my background in C and C++ programming languages, widely used programming models such as pthreads and OpenMP, my experience of designing and implementing parallel algorithms are a high match to the requirements of the job posting.
	   This is the reason, I request you to consider my application.
	   Please let me know if there are any other materials or information that will assist you in processing my application.
	   Thank you for your consideration. I look forward to hearing from you.
%
\end{newlfm}
\end{document}
