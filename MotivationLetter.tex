	% LaTeX file for resume 
% This file uses the resume document class (res.cls)

\documentclass[10pt,stdletter,dateno]{newlfm}
%\usepackage{kpfonts}
\usepackage{url}
\usepackage{hyperref}
\hypersetup{
colorlinks=true,
linkcolor=blue,
filecolor=magenta,      
urlcolor=blue,
}
\usepackage{csquotes}

\widowpenalty=1000
\clubpenalty=1000

\newlfmP{headermarginskip=20pt}
\newlfmP{sigsize=50pt}
\newlfmP{dateskipafter=20pt}
\newlfmP{addrfromphone}
\newlfmP{addrfromemail}
\PhrPhone{Phone}
\PhrEmail{Email}

\namefrom{Rahulkumar\ Gayatri}
\addrfrom{%
    \today\\[10pt]
601, Dream Home Palace\\
Hyderabad, Telangana\\
India, 500035
}
\phonefrom{+91-7899423040}
\emailfrom{rahulgayatri84@gmail.com}

\greetto{To Whom It May Concern,}
\closeline{Sincerely,}
\begin{document}
\begin{newlfm}
	   This is a motivation letter for a position in the Argo H2020 project at INRIA.
	   My qualifications include a Ph.D. Degree from Universitat Politecnica de Catalunya, Barcelona, Spain. 
	   During this period I was a part of the Programming Models group at the Barcelona Supercomputing Center (BSC). 
	   My advisors were Rosa M. Badia and Eduard Ayguade. 
	   My Doctoral thesis is focused on speculative synchronization techniques for the StarSs framework, a task-based programming model.
	   My work included the development of StarSs compiler and runtime features that handle synchronization of multiple threads.
	   I have extensively worked with the Transactional Memory framework to achieve speculative synchronization of threads in StarSs.
	   As a part of the programming models group, I was also involved in the design and implementation of parallel algorithms that use the StarSs framework.
	   Currently I am working as a Technical Specialist, High Performance Computing Group of Wipro Infotech. 
%
\par
	   My work has been published in \textit{EuroPar 2012} and \textit{HiPC 2013} conferences as well as in \textit{MULTIPROG, 2015} workshop.  
	   I am also the co-author of a journal paper published in \textit{Microprocessors and Microsystems, 2014} which explains the work accomplished as a part of the Teraflux project which was sponsored and supported by the European Union. 
	   My resume contains the information of all my publications and their references. 
%
\par
	   My Current work is along the same lines as my research at BSC. 
	   I provide technical assistance to clients who wish to parallelize their code and take advantage of their multicore infrastructure. 
	   Depending on their requirement, I collaborate with them on the design, implementation or on both to help them improve the performance of their application/algorithm.
	   Currently I am working on a project called Moose, which involves simulation of neurons in human brain. 
	   The project was designed and implemented at the National Center for Biological Sciences, India (NCBS). 
	   The multiscale characteristic of this project allows both distributed and shared memory parallelization to be introduced to speedup the simulation. 
	   Currently, I am focussed on the thread-based implementation using pthreads, where each thread updates a part of the cell structure at every timestep. 
	   Later on, the project will include simultaneous updates of different cells on different nodes, which in turn will use threaded parallelism. 
\par
	   My background in mathematics helps me develop novel ideas for parallelization of algorithms and efficient ways to implement them. 
	   That combined with my knowledge of widely used parallel programming models makes my skill set match the requirements of the current job posting.
	   Hence, I would like to apply for this position at INRIA, which is one of the leading research centers in the field of High Performance Computing. 
%
\par
Please let me know if there are any other materials or information that will assist you in processing my application. 
Thank you for your consideration. I look forward to hearing from you.
%
%	   The multidisciplinary and challenging area of parallel computing has taught me to find innovative and efficient solutions to a given problem. 
%	   Work in R\&D did not only provide me with an in depth-knowledge in the areas of mathematics, engineering and CS, but has also given me a valuable insight in the areas of problem analysis, solution design and rapid prototyping.
%	   As a part of the programming models group at Barcelona Supercomputing Center (\url{http://www.bsc.es/computer-sciences/programming-models}), I have also learned how to effectively collaborate with other engineers to achieve the set goals. 
%	   I am also a recipient of the FI-AGAUR scholarship from the Catalan government which is a grant given to encourage research.  
%
\end{newlfm}
\end{document}
