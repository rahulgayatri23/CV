	% LaTeX file for resume 
% This file uses the resume document class (res.cls)

\documentclass[10pt,stdletter,dateno]{newlfm}
%\usepackage{kpfonts}
\usepackage{url}
\usepackage{hyperref}
\hypersetup{
colorlinks=true,
linkcolor=blue,
filecolor=magenta,      
urlcolor=blue,
}
\usepackage{csquotes}

\widowpenalty=1000
\clubpenalty=1000

\newlfmP{headermarginskip=20pt}
\newlfmP{sigsize=50pt}
\newlfmP{dateskipafter=20pt}
\newlfmP{addrfromphone}
\newlfmP{addrfromemail}
\PhrPhone{Phone}
\PhrEmail{Email}

\namefrom{Rahulkumar\ Gayatri}
\addrfrom{%
    \today\\[10pt]
Carrer Consell de cent, 391,4,2\\
Barcelona, Catalonia\\
Spain, 08009
}
\phonefrom{+34-699690160}
\emailfrom{rahulgayatri84@gmail.com}

%\addrto{%
%Carrer Consell de cent, 391,4,2\\
%Barcelona, Catalonia\\
%Spain, 08009}

\greetto{To Whom It May Concern,}
\closeline{Sincerely,}
\begin{document}
\begin{newlfm}

I am writting to apply for the position of \enquote{imec}. 
I received my Ph.D.\ degree from Universitat Politècnica de Catalunya, Spain, in March, 2015. My advisors are Rosa M.\ Badia and Eduard Ayguadè. In my PhD, I have worked extensively with Parallel Programming Models and on the topic of Speculation using Transactional Memory. 
%
\par
In my graduate work, I focus on speculation techniques to synchronize multiple threads of execution in a parallel processor. My work has been published in \textit{EuroPar 2012} and \textit{HiPC 2013} conferences as well as in \textit{MULTIPROG, 2015} workshop.  
I am also the co-author of a journal paper published in \textit{Microprocessors and Microsystems, 2014} which explains the work accomplished as a part of the Teraflux project which was sponsored and supported by the European Union. 
My resume contains the information of all my publications and their references. 
%
\par
The multidisciplinary and challenging area of parallel computing has taught me to find innovative and efficient solutions to a given problem. 
Work in R\&D did not only provide me with an in depth-knowledge in the areas of mathematics, engineering and CS, but has also given me a valuable insight in the areas of problem analysis, solution design and rapid prototyping.
As a part of the programming models group at Barcelona Supercomputing Center (\url{http://www.bsc.es/computer-sciences/programming-models}), I have also learned how to effectively collaborate with other engineers to achieve the set goals. 
I am also a recipient of the FI-AGAUR scholarship from the Catalan government which is a grant given to encourage research.  
%
\par
Rahul would like to use the analytical and problem solving skills that I have acquired to identify research problems and find elegant solutions for them. 
My objective is to use my experience in parallel programming along with my knowledge of the compiler and runtime development to overcome performance bottlenecks in applications. 
%
%After my graduation, for my professional career, I want to use the analytical and problem solving skills acquired in academia to find optimal solutions to real-life problems. 
%My objective is to explore career opportunities in diverse fields and apply there the knowledge gained from my education. 
%
\par
Please let me know if there are any other materials or information that will assist you in processing my application. 
Thank you for your consideration. I look forward to hearing from you.

%\section{Publications}\\
%%Enumerating papers so as to reference later in projects. 
%%
%1] \enquote{Loop level speculation in a task based programming model.} \\
%        \url{http://ieeexplore.ieee.org/xpl/login.jsp?tp=&arnumber=6799132&url=http\%3A\%2F\%2Fieeexplore.ieee.org\%2Fxpls\%2Fabs\_all.jsp\%3Farnumber\%3D6799132}\\\\
%%
%2] \enquote{Transactional access to shared memory in StarSs, a task based programming model.} \\
%        \url{http://link.springer.com/chapter/10.1007/978-3-642-32820-6\_51#page-1} \\\\
%%
%3] \enquote{Analysis of the overheads incurred due to speculation in a task based programming model.} \\
%        \url{http://research.ac.upc.edu/multiprog/papers/multiprog-2015-3.pdf} \\\\
%%
%4] \enquote{TERAFLUX: Harnessing dataflow in next generation teradevices.} \\
%        \url{http://www.sciencedirect.com/science/article/pii/S0141933114000490}
%
\end{newlfm}
\end{document}
