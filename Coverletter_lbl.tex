	% LaTeX file for resume 
% This file uses the resume document class (res.cls)

\documentclass[10pt,stdletter,dateno]{newlfm}
%\usepackage{kpfonts}
\usepackage{url}
\usepackage{hyperref}
\hypersetup{
colorlinks=true,
linkcolor=blue,
filecolor=magenta,      
urlcolor=blue,
}
\usepackage{csquotes}

\widowpenalty=1000
\clubpenalty=1000

\newlfmP{headermarginskip=20pt}
\newlfmP{sigsize=50pt}
\newlfmP{dateskipafter=20pt}
\newlfmP{addrfromphone}
\newlfmP{addrfromemail}
\PhrPhone{Phone}
\PhrEmail{Email}

\namefrom{Rahulkumar\ Gayatri}
\addrfrom{%
    \today\\[10pt]
601, Dream Home Palace\\
Hyderabad, Telangana\\
India, 500035
}
\phonefrom{+91-7899423040}
\emailfrom{rahulgayatri84@gmail.com}

\greetto{To Whom It May Concern,}
\closeline{Sincerely,}
\begin{document}
\begin{newlfm}
	   This is a cover letter for the position of {\it NERSC Exascale Science Applications Postdoctoral Fellow (NESAP)}, Job ID: 81356. 
	   My name is Dr. Rahulkumar Gayatri and I am currently based in Bangalore, India. 
	   I work as a technical specialist in the High Performance Computing group (HPC) at Wipro Infotech. 
	   I have been in this position since September 2015. 
	   I provide technical assistance to clients who wish to parallelize their application/algorithm.
	   For this, I learn the theory and the science involved in the client's algorithm and apply my knowledge of parallel programming to develop parallel versions of the same.
%	   
	   \par
	   Currently I am working on a project called Moose. It involves simulation of neural interactions in a human brain. 
	   The project is designed and implemented at the National Center for Biological Sciences, India (NCBS). 
	   The software uses a set of linear solvers to calculate the chemical and electrical interactions between the neurons. 
	   It is a multi-scale project and allows both distributed and shared memory parallelization to be introduced to speedup the simulation. 
	   The parallelization approach chosen was to implement distributed parallelism (using MPI) to divide the work performed on various cells among the available nodes. 
	   Within each node shared memory parallelism (using OpenMP and Pthreads) is implemented so that each thread updates a part of the cell structure at finite time steps.
	   The shared memory parallelization gave a performance boost of 5X of 8 threads. 
	   Currently I am working on the MPI-based implementation wherein different cells are distributed over the available nodes for simultaneous execution. 
%	   
	   \par
	   Previous to this, I was a Doctoral student at Barcelona Supercomputing Center (BSC), Barcelona, Spain. I graduated in March, 2015. 
	   My advisors were Rosa M.Badia and Eduard Ayguade.  
	   During this period, I was a part of the Programming Models group at the Barcelona Supercomputing Center (BSC). 
	   My Doctoral thesis was focused on speculative synchronization techniques for the StarSs framework, a task-based programming model.
%
	   \par
	   I extended the StarSs framework to speculatively update shared memory locations using STM, instead of the traditional lock and mutex based mechanisms.
	   The results achieved with this idea and implementation showed an increase in performance of applications which have high contention for locks. 
%	   Also the speculative nature of STM increases the available parallelism in an application. 
%
	   \par
	   An extension to speculative memory updates is the speculative execution of tasks, where tasks can be scheduled before their presence in the execution flow can be confirmed.
	   I implemented a lighweight rollback mechanism which can be used to undo the updates of tasks in case the speculation fails.
	   The idea of greedy task execution, wherein the tasks are scheduled even before their validity can be ascertained improved the performance by an average of 20\%. 
%
	   \par
	   At BSC, I also worked on porting applications using SMPSs, StarSs implementation for SMPs onto Symmetric Multiprocessors (SMPs).
	   I built applications such as NQueens, Specfem, Jacobi and Gauss-Seidel solvers, that show the performance benefits, ease of use and portability of using the StarSs framework. 
	   I also parallelized the Graph500 benchmark suite using SMPSs.
	   My background in mathematics helps me develop novel ideas for parallelization of algorithms and efficient ways to implement them. 
	   These applications are a part of the StarSs application repository. 
%
	   \par
	   In my Mtech thesis, I designed and implemented a Breadth First Search algorithm, that optimises the use of low memory available in the Synergistic Processing Element(SPE) of IBM's Cell.B.E processor. 
	   The strong point of the implementation was this design of the data structure to represent a node, which increased the memory locality.
%	   \par
%	   My doctoral work has been published in \textit{EuroPar 2012} and \textit{HiPC 2013} conferences as well as in \textit{MULTIPROG, 2015} workshop.  
%	   I am also the co-author of a journal paper published in \textit{Microprocessors and Microsystems, 2014} which explains the work accomplished as a part of the Teraflux project which was sponsored and supported by the European Union. 
%	   I have 2 posters which highlight my work on parallelization of applications. 
%	   My resume contains the information of all my publications. 
%
	   \par
	   Working in the programming models group at BSC and in the TERAFLUX project has taught me to effectively collaborate with other researchers to achieve the set goals.
	   The above mentioned details highlight my experience in development and implementation of parallel algorithms from different scientific domains.
	   I am also experienced in the use of various tools to analyze the results obtained and optimize on the bottlenecks.
	   I have experience in working with the widely used parallel programming models and work collaboratively within a team.
	   Hence, I believe my skill set is an ideal match for the current job posting. 
	   I feel the advances in High Performance Computing area and parallel programming should be applied to other scientific domains in order to further understand the mysteries of nature.
	   This is my main motivation to apply for the current job. 
	   Also, I am highly interested in working at the prestigious organization such as the Lawrence Berkeley National Laboratory and hope that you will consider my application for further processinng. 
%
	   \par
	   The information regarding my publications is available in my resume.
	   Please let me know if there are any other materials or information that will assist you in processing my application.
	   Thank you for your consideration. I look forward to hearing from you.
%
\end{newlfm}
\end{document}
